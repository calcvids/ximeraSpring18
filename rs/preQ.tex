\documentclass[handout]{ximera}

\title{Post-Video Questions Preview}

\begin{document}

\begin{abstract}
\end{abstract}

% set 4

\maketitle

Here are some questions you’ll be asked after you finish watching the video. Please read through these before watching the video.

\begin{problem}
Coal gas is produced at a gasworks. Pollutants in the gas are removed by scrubbers, which become less and less efficient as time goes on. The following measurements, made twice each month, show the rate at which pollutants are escaping (in tons/month) in the gas. Make an underestimate for the total quantity of pollutants that escape during the three-month time period shown here.

\begin{table}[h!]
\centering
\caption{Time and Rate}
\label{my-label2}
\begin{tabular}{|c|c|c|c|c|c|c|c|}
\hline
Time (months)     & 0 & 0.5 & 1  & 1.5 & 2  & 2.5 & 3  \\
\hline
Rate (tons/month) & 2 & 6   & 10 & 17  & 27 & 38  & 50 \\
\hline
\end{tabular}
\end{table}

\end{problem}

\begin{problem}
The rate (in liters per minute) at which water drains from a tank is recorded at half-minute intervals. Use the data below to make an overestimate for $\int_0^3 R(t) dt$.
\begin{table}[h!]
\centering
\caption{Time and Rate}
\label{my-label3}
\begin{tabular}{|c|c|c|c|c|c|c|c|}
\hline
t (minutes)          & 0  & 0.5 & 1  & 1.5 & 2  & 2.5 & 3  \\
\hline
Rate (liters/minute) & 40 & 38  & 36 & 34  & 32 & 30  & 28 \\
\hline
\end{tabular}
\end{table}

\end{problem}



\end{document}




