\documentclass[handout]{ximera}

\title{Main Ideas}

\begin{document}

\begin{abstract}
\end{abstract}

\maketitle

Here are the main points that are addressed in the video. Please read these and think about them as you watch.

\begin{itemize}
\item If you know the rate at which something accumulates, a Riemann Sum tells you the total amount that accumulates
\item In a Riemann Sum, you break things down into small intervals, compute the amount of accumulation for each interval, and then add these amounts together to get the total
\item In each interval, you assume that the rate of accumulation is constant, and multiply that rate by the size of the interval
\end{itemize}

\end{document}