\documentclass[handout]{ximera}

\title{Post Video Questions}

\begin{document}

\begin{abstract}
\end{abstract}

% set 7 post video

\maketitle

\begin{javascript}
post1 = function(a) {
    return a==50 || a==74 || a==62;
  };
\end{javascript}

Please answer each of these questions to the best of your ability. You are welcome to re-watch parts of any of the video to help you.

\begin{problem}
Coal gas is produced at a gasworks. Pollutants in the gas are removed by scrubbers, which become less and less efficient as time goes on. The following measurements, made twice each month, show the rate at which pollutants are escaping (in tons/month) in the gas. Make an underestimate for the total quantity of pollutants that escape during the three-month time period shown here.

\begin{table}[h!]
\centering
\caption{Time and Rate}
\label{my-label2}
\begin{tabular}{|c|c|c|c|c|c|c|c|}
\hline
Time (months)     & 0 & 0.5 & 1  & 1.5 & 2  & 2.5 & 3  \\
\hline
Rate (tons/month) & 2 & 6   & 10 & 17  & 27 & 38  & 50 \\
\hline
\end{tabular}
\end{table}

$\answer[validator=post1]{}$
\end{problem}

\begin{problem}
The rate (in liters per minute) at which water drains from a tank is recorded at half-minute intervals. Use the data below to make an overestimate for $\int_0^3 R(t) dt$.
\begin{table}[h!]
\centering
\caption{Time and Rate}
\label{my-label3}
\begin{tabular}{|c|c|c|c|c|c|c|c|}
\hline
t (minutes)          & 0  & 0.5 & 1  & 1.5 & 2  & 2.5 & 3  \\
\hline
Rate (liters/minute) & 40 & 38  & 36 & 34  & 32 & 30  & 28 \\
\hline
\end{tabular}
\end{table}

$\answer{105}$
\end{problem}


\end{document}
