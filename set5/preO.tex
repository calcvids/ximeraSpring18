\documentclass[handout]{ximera}

\title{Main Ideas}

\begin{document}

\begin{abstract}
\end{abstract}

\maketitle

Here are the main points that are addressed in the video. Please read these and think about them as you watch.


\begin{itemize}
\item $y = f(g(x))$ is a composite function. This means that the outputs of the function $g$ become the inputs of the function $f$.
\item The derivative of the function $y = f(x)$ at $x = a$ conveys how many times as large a very small change in $y$ is compared to the corresponding small change in $x$ away from $x = a$.
\item Developing a method for computing the derivative of the composite function $y = f(g(x))$ requires determining how much $f(g(x))$ changes when $x$ changes by a very small amount.
\item The derivative of the composite function $y = f(g(x))$ is $f'(g(x)) \times g'(x)$.
\end{itemize}


\end{document}