\documentclass[handout]{ximera}

\title{Post-Video Questions Preview}

\begin{document}

\begin{abstract}
\end{abstract}

% set 2

\maketitle

Here are some questions you’ll be asked after you finish watching the videos. Please read through these before watching the videos.

A SUV monitors its fuel consumption as it starts to drive along a hilly highway. The SUV’s internal computer records the level of fuel in its tank and its odometer readings periodically, shown in the table below:
\begin{table}[h!]
\centering
\begin{tabular}{ll}
\hline
Fuel Level (gallons) & Distance (miles) \\
9.7                  & 2                \\
9.6                  & 5                \\
9.5                  & 8                \\
9.3                  & 12               \\
8.9                  & 14               \\
8.6                  & 16               \\
8.4                  & 18               \\
8.3                  & 21               \\
\end{tabular}
\end{table}

Compute an underestimate of the SUV’s fuel economy when there are 8.9 gallons of fuel in its tank, assuming that the rate of fuel consumption is either constantly increasing or constantly decreasing.

Briefly explain why the value you computed is an underestimate.

What information would you need to improve your approximation of the SUV’s fuel economy at the 8.9-gallon mark?


\end{document}




