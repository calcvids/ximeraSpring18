\documentclass[handout]{ximera}

\title{Video Set Introduction}

\begin{document}

\begin{abstract}
\end{abstract}

%Video: Intro to Approximating Speed %Calculus Video 1

\maketitle

\begin{javascript}
  nameCheck = function(a,b) {
    return a.toLowerCase() != b.toLowerCase();
  };
\end{javascript}

Before watching the videos, think about and answer these questions to the best of your ability. Your answer will always be recorded as correct, regardless of your answer choice.

A car speeds up as it drives away from a traffic light. The car’s GPS unit records its distance from the light in the table below:


\begin{table}[h!]
\centering
\caption{Time and Distance}
\label{my-label}
\begin{tabular}{ll}
Time (seconds) & Distance (meters) \\
0              & 0                 \\
1              & 1                 \\
2              & 3                 \\
3              & 6                 \\
4              & 10                \\
5              & 15                \\
6              & 21                \\
7              & 27
\end{tabular}
\end{table}

\begin{problem}
Compute an approximation of the car’s speed at the 5-second mark.
$\answer[format=string,validator=nameCheck]{}$.
\end{problem}

\begin{problem}
Is the value you computed:
\begin{multipleChoice}
\choice[correct]{Equal to the car’s speed at the 5-second mark}
\choice[correct]{An underestimate of the car’s speed at the 5-second mark}
\choice[correct]{An overestimate of the car’s speed at the 5-second mark}
\choice[correct]{Neither an underestimate nor an overestimate}
\choice[correct]{You can’t tell without having more information}
\end{multipleChoice}
\end{problem}

\begin{problem}
How could you improve your approximation of the car’s speed at the 5-second mark?
\begin{multipleChoice}
\choice[correct]{You don’t need to make an improvement because the speed you calculated \emph{is} the car’s speed at the 5-second mark}
\choice[correct]{Use a different pair of points from the table to compute the speed}
\choice[correct]{Use two pairs of points from the table to compute two speeds, and then average these speeds}
\choice[correct]{Use a larger interval of time (e.g., if you originally used a 1-second time interval, a 2-second time interval would improve your approximation)}
\choice[correct]{Use a smaller interval of time (which would require additional information)}
\end{multipleChoice}
\end{problem}



\end{document}



\begin{problem}
The graph below represents the relationship between a car’s distance in kilometers from an intersection (represented by $f(t)$) and the number of minutes elapsed since the car passed the intersection (represented by the variable $t$). Approximate the average rate of change of $f(t)$ with respect to t over the interval $[4, 6]$.

[graph here]

Rate of change: $\answer{5}$

\end{problem}

\begin{problem}
A car is driving away from a traffic light. The distance $d$ (in feet) of the car from the traffic light $t$ seconds since the car started moving is given by the formula $d = 1.3t^2 - 17$.
Write an expression that represents the approximate the speed of the car 5 seconds after it started moving.
$d(t) = \answer{5}$\\

Is the value of the expression you wrote in part an overestimate or underestimate of the car’s actual speed 5 seconds after it started moving?

\begin{multipleChoice}
\choice[correct]{overestimate}
\choice[correct]{underestimate}
\end{multipleChoice}

\end{problem}