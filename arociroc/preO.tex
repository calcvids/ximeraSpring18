\documentclass[handout]{ximera}

\title{Main Ideas}

\begin{document}

\begin{abstract}
\end{abstract}

\maketitle

Here are the main points that are addressed in the video. Please read these and think about them as you watch.

\begin{itemize}
\item In calculus, we often refer to the \textit{instantaneous rate of change} of one quantity with respect to another. This is slightly misleading since a rate of change is a comparison between changes in quantities’ measures. Rates of change do not occur at an instant - \textit{they require CHANGES in quantities’ measures to exist}.
\item Because instantaneous rate of change cannot be measured directly, it must be approximated using average rates of change.
\item We can often improve the accuracy of approximations of an instantaneous rate of change by decreasing the size of the interval over which we compute the average rate of change approximation.
\end{itemize}

\end{document}