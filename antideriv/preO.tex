\documentclass[handout]{ximera}

\title{Main Ideas}

\begin{document}

\begin{abstract}
\end{abstract}

\maketitle

Here are the main points that are addressed in the video. Please read these and think about them as you watch.

\begin{itemize}
\item An antiderivative of a function $f(x)$ is a functions whose derivative is $f(x)$
\item The \textit{general antiderivative} represents all the functions whose derivatives are $f(x)$. If $F(x)$ is an antiderivative to $f(x)$, meaning $F'(x)=f(x)$, then $F(x)+C$ represents all the functions whose derivatives are $f(x)$, where $C$ represents any constant.
\item The reason for the term $C$ is that the derivative of a constant is 0, so adding a (positive or negative) constant to a function will not change the derivative, so the derivative of $F(x)+C$ is the same as the derivative of $F(x)$.
\end{itemize}
\end{document}