\documentclass[handout]{ximera}

\title{Main Ideas}

\begin{document}

\begin{abstract}
\end{abstract}

\maketitle

Here are the main points that are addressed in the video. Please read these and think about them as you watch.

\begin{itemize}
\item When doing optimization problems it is important to distinguish between measurable quantities that vary and quantities that do not vary, often labeling those measurable quantities that vary with variable names.
\item Identify the quantity to be optimized (such as area), any formula related to this quantity (such as $A=lw$) and any constraints (the perimeter is 300). When needed, use the constraints to eliminate a variable in the formula, rewriting the formula as a function in terms of only one variable.
\item Candidates for maximum and minimum values occur at critical points and endpoints (when appropriate).
\item Use a derivative test to verify if a candidate indeed yields a maximum or minimum value.
\end{itemize}

\end{document}