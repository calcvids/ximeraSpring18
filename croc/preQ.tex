\documentclass[handout]{ximera}

\title{Post-Video Questions Preview}

\begin{document}

\begin{abstract}
\end{abstract}

%Video: Intro to Approximating Speed %Calculus Video 1

\maketitle

Here are some questions you’ll be asked after you finish watching the videos. Please read through these before watching the videos.

\begin{problem}
Suppose Jason is sailing his boat straight across a lake at a constant rate of 0.12 meters/second.

\begin{enumerate}
\item Let $\Delta t$ represent a change in the number of seconds elapsed during some part of Jason’s ride and let $\Delta d$ represent the corresponding change in the number of meters Jason traveled. Write an equation that expresses the relationship between $\Delta t$ and $\Delta d$.
\item Jason passes an island while traveling at this constant rate. At 10:30 AM, Jason is 3 meters past the island. At what time did Jason pass the island?
\end{enumerate}
\end{problem}

\begin{problem}
Imagine you are driving on the highway and vary your speed to maintain a constant fuel efficiency. Select the choices to complete the statement to most accurately capture what it means to drive with a constant fuel economy:
For [1] [2], the [3] is [4].

[1]: fixed, increasing, decreasing\\

[2]: gallons, distance, amount of change in gallons, amount of change in distance\\

[3]: gallons, distance, amount of change in gallons, amount of change in distance\\

[4]: constant, increasing, decreasing\\

\end{problem}

\begin{problem}
Suppose $x$ and $y$ represent the measures of two quantities and $y$ changes at a constant rate of $-0.9$ with respect to $x$. As $x$ changes from 7 to 9.5, how much does $y$ change?
\end{problem}






\end{document}